\documentclass{osuthesis}
\usepackage{graphicx}
\usepackage{multirow}

\usepackage{listings}
\usepackage{color}
\usepackage{times}
\definecolor{dkgreen}{rgb}{0,0.6,0}
\definecolor{gray}{rgb}{0.5,0.5,0.5}
\definecolor{mauve}{rgb}{0.58,0,0.82}

\lstset{
	language=Java,
	aboveskip=3mm,
	belowskip=3mm,
	showstringspaces=false,
	columns=flexible,
	basicstyle={\small\ttfamily},
	numbers=none,
	numberstyle=\tiny\color{gray},
	keywordstyle=\color{blue},
	commentstyle=\color{dkgreen},
	stringstyle=\color{mauve},
	breaklines=true,
	breakatwhitespace=true,
	tabsize=3
	}


\graphicspath{ {images/} }
              \title{HADOOP IMAGE PROCESSING FRAMEWORK} % Title in ALL CAPS
             \formattedtitle{HADOOP IMAGE PROCESSING FRAMEWORK} % Same as \title, but with line
                                       % breaks to obtain the inverted
                                      % pyramid shape required for
                                      % title and approval pages
              \author{SRIDHAR VEMULA} % Author in ALL CAPS

             % Earned degree(s)
              \degreeone{\\%
               Master of Science in Computer Science\\%
               Oklahoma State University\\%
               Stillwater, OK\\%
               2015}
% %             \degreeone{%
%               <Degree>\\%
 %               <Institution>\\%
  %              <City>, <State/Country>\\%
   %             <Year>} % (If applicable)

              % Current degree information
              \degreesought{MASTER OF SCIENCE} % Degree in ALL CAPS
              \degreedate{May, 2015}
              \majorfield{Computer Science}

              \begin{document}
             \bibliographystyle{plain}  % Several styles available,
                                         % check with your department

             % FOREMATTER
              \maketitle        % Creates title page.
              \makeapproval{3}  % Creates spproval page.  The argument
                                % (the numeral 4,5, or 6) is the number of
                               % signatures required (Remember to add one
                                % for the Dean of the Graduate College.)
%             \begin{preface}   % Creates Preface page (OPTIONAL)
             %   <Preface text here.>
 %             \end{preface}

  %            \begin{acknowledge}  % Creates Acknowledgments
                                   % page (OPTIONAL)
               % <Acknowledgment text here.>
   %          \end{acknowledge}
			 \begin{abstract}{SRIDHAR VEMULA}{MASTER IN SCIENCE}{COMPUTER SCIENCE} % Creates abstract
			              \par With the rapid growth of social media, the number of images being
			              uploaded to the internet is exploding. Massive quantities of images
			              are shared through multi-platform services such as Snapchat,
			              Instagram, Facebook and WhatsApp; recent studies estimate that over
			              1.8 billion photos are uploaded every day.  However, for the most
			              part, applications that make use of this vast data have yet to
			              emerge. Most current image processing applications, designed for
			              small-scale, local computation, do not scale well to web-sized
			              problems with their large requirements for computational resources and
			              storage.  The emergence of processing frameworks such as the Hadoop
			              MapReduce\cite{dean2008} platform addresses the problem of providing
			              a system for computationally intensive data processing and distributed
			              storage. However, to learn the technical complexities of developing
			              useful applications using Hadoop requires a large investment of time
			              and experience on the part of the developer.  As such, the pool of
			              researchers and programmers with the varied skills to develop
			              applications that can use large sets of images has been limited. To
			              address this we have developed the Hadoop Image Processing Framework,
			              which provides a Hadoop-based library to support large-scale image
			              processing. The main aim of the framework is to allow developers of
			              image processing applications to leverage the Hadoop MapReduce
			              framework without having to master its technical details and introduce
			              an additional source of complexity and error into their programs.

			              \end{abstract}
              \tableofcontents  % Creates table of contents
              \listoftables
             \listoffigures    % Creates list of figures (IF APPLICABLE)
	       
%              \listoftables     % Creates list of tables (IF APPLICABLE)
			
              % BODY OF THESIS
             % \include{body2}  % (Recommend storing text in separate *.tex file, then
                      % including with \include{<filename>}.)
				
				
				\chapter{INTRODUCTION}
				With the spread of social media in recent years, a large amount of
				image data has been accumulating. When processing this massive data
				resource has been limited to single computers, computational power and
				storage ability quickly become bottlenecks. Alternately, processing
				tasks can typically be performed on a distributed system by dividing
				the task into several subtasks. The ability to parallelize tasks
				allows for scalable, efficient execution of resource-intensive
				applications.  The Hadoop MapReduce framework provides a platform for
				such tasks.
				
				When considering operations such as face detection, image
				classification\cite{Li2009} and other types of processing on images,
				there are limits on what can be done to improve performance of single
				computers to make them able to process information at the scale of
				social media. Therefore, the advantages of parallel distributed
				processing of a large image dataset by using the computational
				resources of a cloud computing environment should be considered. In
				addition, if computational resources can be secured easily and
				relatively inexpensively, then cloud computing is suitable for
				handling large image data sets at very low cost and increased
				performance. Hadoop, as a system for processing large numbers of
				images by parallel and distributed computing, seems promising. In
				fact, Hadoop is in use all over the world. Studies using Hadoop have
				been performed, dealing with text data files\cite{Lin2010}, analyzing
				large volumes of DNA sequence data\cite{McKenna2010}, converting the
				data of a large number of still images to PDF format, and carrying out
				feature selection/extraction in astronomy\cite{wiley2011}.  These
				examples demonstrate the usefulness of the Hadoop system, which can
				run multiple processes in parallel for load balancing and task
				management.
				
				Most of the image processing applications that use the Hadoop
				MapReduce framework are highly complex and impose a staggering
				learning curve.  The overhead, in programmer time and expertise,
				required to implement such applications is cumbersome.  To address
				this, we present the Hadoop Image Processing Framework, which hides
				the highly technical details of the Hadoop system and allows
				programmers who can implement image processing algorithms but who have
				no particular expertise in distributed systems to nevertheless
				leverage the advanced resources of a distributed, cloud-oriented
				Hadoop system. Our framework provides users with easy access to
				large-scale image data, smoothly enabling rapid prototyping and
				flexible application of complex image processing algorithms to very
				large, distributed image databases.
				
				The Hadoop Image Processing Framework is largely a software
				engineering platform, with the goal of hiding Hadoop's complexity
				while providing users with the ability to use the system for
				large-scale image processing without becoming crack Hadoop engineers.
				The framework's ease of use and Java-oriented semantics will further
				ease the process of creating large scale image applications and
				experiments. This framework is an excellent tool for novice Hadoop
				users, image application developers and computer vision researchers,
				allowing the rapid development of image software that can take
				advantage of the huge data stores, rich metadata and global reach of
				current online image sources.%\footnote[1]%{The Hadoop Image Processing
					%Framework is open source and freely available for download at
					%https://github.com/okstate-robotics/hipl.git}.
				
				In the following section we will describe prior work in this
				area. Next we present an overview of the Hadoop Image Processing
				Framework including the Downloader, Processor and Extractor
				stages. Additionally, we describe our approach of distributing tasks
				for MapReduce. Finally, we demonstrate the potential of this framework
				with quantitative analysis and experiments performed on image
				processing tasks.
				\chapter{LITERATURE REVIEW}
				\section{Background}
				\par Apache Hadoop \cite{Hadoop2009} is a platform that offers an efficient and effective method of storing and processing massive amounts of data. Unlike traditional offerings, Hadoop was designed and built from ground up to address the requirements and challenges of big data. Apache Hadoop use cases are many and show up in many industries, including : risk, fraud and portfolio analysis in financial services; behavior analysis and personalization in retail; social network, relationship and sentiment analysis for marketing; drug interaction modeling and genome data processing in health care and life sciences … to name a few.
				\par At its core, Apache Hadoop is a frame work for scalable and reliable distributed data storage and processing. It allows for the processing of large datasets across clusters of computers using a simple programming model. It is designed to scale up from single server to thousands of machines, aggregating the local computation and storage form each server., Rather than relying on expensive hardware, the Hadoop software detects and handles any failures that occur, allowing to achieve high availability on top of inexpensive commodity computes, each individually prone to failure.
				At the core of Apache Hadoop are the Hadoop Distributed File System or HDFS and Hadoop MapReduce, which provides a framework for distributed processing.
				\par \textbf{HDFS :} The Hadoop Distributed File System \cite{Shvachko2010} is a scalable and reliable distributed storage system that aggregates the storage of every node in a Hadoop cluster into single global file system. HDFS stores individual files in large blocks, allowing it to efficiently store very large ot numerous files across multiple machines and access individual chunks of data in parallel, without needing to read the entire file into a single computers memory. Reliability is achieved by replicating the data across multiple hosts, with each block of data being stored, by default, on three separate computers. If an individual node fails, the data remains available and an additional copy of any blocks it holds may be made on new machines to protect against failures. This approach allows HDFS to dependably store massive amounts of data.
				\par \textbf{MapReduce :} MapReduce \cite{dean2008} is a programming model that allows Hadoop to efficiently process large amounts of data. MapReduce breaks large data processing problems into multiple steps, namely Maps and Reduces that can each be worked on at the same time on multiple computers. MapReduce programs are designed to compute large volumes of data in parallel fashion. This requires dividing the workload across large number of machines. This model would not scale to large clusters if the components were allowed to share data arbitrarily. The communication overhead required to keep data on the nodes synchronized at all times would prevent the system from performing reliably and efficiently at large scale. Instead all data elements in MapReduce are immutable, meaning they cannot be updated. If in mapping task you change an input(key, value) pair, it doesn’t get reflected back in the input files; communication occurs only by generating new output(key, value) pairs which are forwarded by Hadoop system into the next phase of execution. Conceptually, MapReduce programs transform lists of input data elements into lists of output data elements. A MapReduce program will do this twice, using two different list processing idioms: map and reduce. The first phase of map reduce program is mapping. A list of data elements are provided, one at a time to a function called Mapper, which transforms each element individually to an output data element. The second phase of map reduce program is Reducing. Reducing lets you aggregate values together. A reducer function receives an iterator of input values form an input list. It then combines these values together, returning a single output value.
				\begin{figure}
				\centering
				    \includegraphics[width=0.90\textwidth]{images/yahoo.png}
				    \caption{Detailed Hadoop MapReduce data flow. Source yahoo}
				    \label{Fig: Detailed Hadoop MapReduce data flow. Source yahoo }
				\end{figure}
				\section{Prior Work}
				With the rapid usage increase of online photo storage and social media
				on sites like Facebook, Flickr and Picasa, more image data is
				available than ever before, and is growing every day.  Every minute
				27,800 photos are uploaded to Instagram,\cite{Horaczek2013} while
				Facebook receives 208,300 photos over the same time frame. This alone
				provides a source of image data that can scale into the billions.  The
				explosion of available images on social media has motivated image
				processing research and application development that can take
				advantage of very large image data stores.
				
				White et.al \cite{White2010} presents a case study of classifying and
				clustering billions of regular images using MapReduce.  It describes
				an image pre-processing technique for use in a sliding-window approach
				for object recognition.  Pereira et.al \cite{Pereira2010} outlines
				some of the limitations of the MapReduce model when dealing with
				high-speed video encoding, namely its dependence on the NameNode as a
				single point of failure, and the difficulties inherent in generalizing
				the framework to suit particular issues.  It proposes an alternate
				optimized implementation for providing cloud-based IaaS
				(Infrastructure as a Service) solutions.  Lv et.al \cite{Lv2010}
				describes using the $k$-means algorithm in conjunction with MapReduce
				and satellite/aerial photographs in order to find different elements
				based on their color.
				
				Zhang et.al \cite{Zhang2010} presents methods used for processing
				sequences of microscope images of live cells. The images are
				relatively small (512x512, 16-bit pixels).  Stored in 90 MB folders,
				the authors encountered difficulties regarding fitting into Hadoop DFS
				blocks with were solved by custom InputFormat, InputSplit and
				RecordReader classes.  Powell et.al \cite{Powell2010} describes how
				NASA handles image processing of celestial images captured by the Mars
				Orbiter and rovers. Clear and concise descriptions are provided about
				the segmentation of gigapixel images into tiles, how the tiles are
				processed and how the image processing framework handles scaling and
				works with the distributed processing. Wang, Yinhai and
				McCleary\cite{Wang2011} discuss speeding up the analysis of tissue
				microarray images by substituting human expert analysis for automated
				processing algorithms. While the images were gigapixel-sized, the
				content was easily segmented and there was no need to analyze all of
				an image at once. The work was all done on a specially-built high
				performance computing platform using the Hadoop framework.
				
				Bajcsy et.al \cite{Bajcsy2013} present a characterization of four
				basic terabyte-size image computations on a Hadoop cluster in terms of
				their relative efficiency according to a modified Amdahl's Law. The
				work was motivated by the fact that there is a lack of standard
				benchmarks and stress tests for large-scale image processing
				operations on the Hadoop framework. Moise et.al \cite{Moise2013}
				outlines the querying of thousands of images in one run using the
				Hadoop MapReduce framework and the eCP Algorithm. The experiment
				performs an image search on 110 million images collected from the web
				using the Grid 5000 platform. The results are evaluated in order to
				understand the best practices for tuning Hadoop MapReduce performance
				for image search.
				
				The above shows that there has been a great deal of intensive work in
				image processing using MapReduce.  However, each independent project
				requires a complex, error-prone, one-off implementation.  Such
				research and application development would benefit greatly from a
				standard, well-engineered image processing framework such as the one
				we provide.
				
				\chapter{METHODOLOGY}
				\textbf{Thesis Statement :}\textit{To design and develop an Image Library using the Hadoop MapReduce framework thereby providing a tool to users, researchers to develop of large scale image processing application with ease.} 
				\newline
				\newline
				The Hadoop Image Processing Framework is intended to provide users
				with an accessible, easy-to-use tool for developing large-scale image
				processing applications.
				
				The main goals of the Hadoop Image Processing Framework are:
				\begin{itemize}
					\item Provide an open source framework over Hadoop MapReduce for
					developing large-scale image applications
					\item Provide the ability to flexibly store images in various Hadoop
					file formats
					\item Present users with an intuitive application programming
					interface for image-based operations which is highly parallelized
					and balanced, but which hides the technical details of Hadoop
					MapReduce
					\item Allow interoperability between various image processing
					libraries
				\end{itemize}
				
				\section{Downloading and storing image data}
				Hadoop uses the Hadoop Distributed File System (HDFS)\cite{Shvachko2010} to store files
				in various nodes throughout the cluster.  One of Hadoop's significant
				problems is that of small file storage. \cite{White2009} A small file
				is one which is significantly smaller than HDFS block size.  Large
				image datasets are made up of small image files in great numbers,
				which is a situation HDFS has a great deal of trouble
				handling. This problem can be solved by providing a container to group
				the files in some way. Hadoop offers a few options:
				\begin{itemize}
					\item HAR File
					\item Sequence File
					\item Map File
				\end{itemize}
				\begin{figure}[h]
					\centering
					\includegraphics[width=0.90\textwidth]{down-node}
					\caption{Single node running the Downloader Module (handled by
						the framework and transparent to the user)}
					\label{fig:down-node}
				\end{figure}
				The Downloader Module of our Hadoop Image Processing Framework
				performs the following operations:
				
				\textbf{Step 1: Input a URL List.} Initially users input a file
				containing URLs of images to download. The input list should be a text
				file with one image URL per line. The list can be generated by hand,
				extracted from a database or a provided by a search. The framework
				provides an extendable ripper module for extracting URLs from Flickr
				and Google image searches and from SQL databases. In addition to the
				list the user selects the type of image bundle to be generated
				(e.g. HAR, sequence or map).  Our system divides the URLs for download
				across the available processing nodes for maximum efficiency and
				parallelism. The URL list is split into several map tasks of equal size across the nodes. Each node map task generates several image bundles
				appropriate to the selected input list, containing all of the image URLs
				to download. In the reduce phase, the Reducer will merge these image
				bundles into a large image bundle.
				
				
				\textbf{Step 2: Split URLs across nodes.}  From the input file
				containing the list of image URLs and the type of file to be
				generated, we equally distribute the task of downloading images across
				the all the nodes in the cluster. The nodes are efficiently managed so
				that no memory overflow can occur even for terabytes of images
				downloaded in a single map task. This allows maximum downloading
				parallelization. Image URLs are distributed among all available
				processing nodes, and each map task begins downloading its respective
				image set.
				
				\begin{figure}[h]
					\centering
					\includegraphics[width=0.90\textwidth]{down-map}
					\caption{Individual map task of Downloader Module}
					\label{fig:down-map}
				\end{figure}
				
				\textbf{Step 3: Download image data from URLs.}  For every URL
				retrieved in the map task, a connection is established according to
				the appropriate transfer protocol (e.g. FTP, HTTP, HTTPS, etc.). Once
				connected, we check the file type.  Valid images are assigned
				InputStreams associated with the connection. From these InputStreams,
				we generate new HImage objects and add the images to the image
				bundle. The HImage class holds the image data and provides an
				interface for the user's manipulation of image and image header
				data. The HImage class also provides interoperability between various
				image data types (e.g. BufferedImage, Mat, etc.).
				
				\textbf{ Step 4: Store images in an image bundle. }  Once an HImage
				object is received, it can be added to the image bundle simply by
				passing the HImage object to the appendImage method. Each map task
				generates a number of image bundles depending on the image list. In
				the reduce phase, all of these image bundles are merged into one large
				bundle.
				
				
				\section{Processing image bundle using MapReduce}
				Hadoop MapReduce program handles input and output data very
				efficiently, but their native data exchange formats are not convenient
				for representing or manipulating image data.  For instance,
				distributing images across map nodes require the translation of images
				into strings, then later decoding these image strings into specified
				formats in order to access pixel information.  This is both
				inefficient and inconvenient. To overcome this problem, images should
				be represented in as many different formats as possible, increasing
				flexibility. The framework focuses on bringing familiar data types
				directly to user.
				
				As distribution is important in MapReduce, images should be processed
				in the same machine where the bundle block resides. In a generic
				MapReduce system, the user is responsible for creating InputFormat and
				RecordReader classes to specify the MapReduce job and distribute the
				input among nodes. This is a cumbersome and error-prone task; the
				Hadoop Image Processing Framework provides such InputFormat and
				RecordReaders for system's ease of use.
				\begin{figure}[h]
					\centering
					\includegraphics[width=0.90\textwidth]{pro-node}
					\caption{Single node running the Processor Module (handled by
						the framework and transparent to user)}
					\label{fig:pro-node}
				\end{figure}
				
				Images are distributed as various image data types and users have
				immediate access to pixel values.  If pixel values are naively
				extracted from the image byte formats, valuable image header data
				(e.g. JPEG EXIF data or IHDR \cite{David03} image headers) is
				lost. The framework holds the image data in a special HImageHeader
				data type before converting the image bytes into pixel values.  After
				processing pixel data, image headers are reattached to the processed
				results.  The small amount of storage overhead required for this
				functionality is a trade-off worth making for preserving
				image header data after processing.
				
				The functionality of the framework's Processor module is described
				below:
				
				\textbf{Step 1: Devise the algorithm.} We assume that the user writes
				an algorithm which extends the provided GenericAlgorithm class. This
				class is passed as an argument to the processor module. The framework
				starts a MapReduce job with the algorithm as an input. The
				GenericAlgorithm holds an HImage variable; this allows user to write
				an algorithm on a single image data structure, which the framework
				then iterates over the entire image bundle. In addition to the
				algorithm, the user should provide the image bundle file that needs to
				be processed.  Depending on the specifics of the image bundle
				organization and contents, the bundle is divided across nodes as
				individual map tasks. Each map task will apply the processing
				algorithm to each local image and append them to the output
				image bundle. In the reduce phase, the Reducer merges these image
				bundles into a large image bundle.
				
				\textbf{Step 2: Split image bundle across nodes.} The input image
				bundle is stored as blocks in the HDFS.  In order to obtain maximum
				throughput, the framework establishes each map task to run in the same
				block where it resides, using custom input format and record reader
				classes. This allows maximum parallelization without the problem of
				transferring data across nodes.  Each image bundle now applies
				different map tasks to the image data for which it is responsible.
				
				\begin{figure}[h]
					\centering
					\includegraphics[width=0.80\textwidth]{pro-map}
					\caption{Individual map task of Processor Module}
					\label{fig:pro-map}
				\end{figure}
				
				\textbf{Step 3: Process individual image.}  The processing algorithm
				devised by the user and provided as input to the Processing Module is
				applied to every HImage retrieved in the map task.  The HImage
				provides its image data in the data format (e.g. Java BufferedImage,
				OpenCV Mat, etc.) requested by the user and used by the processing
				algorithm. Once the image data type is retrieved, processing takes
				place. After processing, the preserved image header data from the
				original image is appended to the processed image. The processed image
				is appended to the temporary bundle generated by the map task.
				
				\textbf{Step 4: Store processed images in an image bundle.} Every map
				task generates an image bundle upon completion of its processing.
				Once the map phase is completed there are many bundles scattered
				across the computing cluster.  In the reduce phase, all of these
				temporary image bundles are merged into a single large file which
				contains all the processed images.
				
				
				\section{Extracting image bundles using MapReduce}
				In addition to creating and processing image bundles, the framework
				provides a method for extracting and viewing these images.  Generally,
				Hadoop extracts images from an image bundle iteratively, inefficiently
				using a single node for the task. To address this inefficiency, we
				designed an Extractor module which extracts images in parallel across
				all available nodes.
				\begin{figure}[h]
					\centering
					\includegraphics[width=0.80\textwidth]{ext-node}
					\caption{Single node running the Extractor Module (handled by the
						framework and transparent to the user)}
					\label{fig:ext-node}
				\end{figure}
				Distribution plays a key role in MapReduce; we want to make effective
				and efficient use of the nodes in the computing cluster.  As
				previously mentioned in the description of the Processor module, a
				user working in a generic Hadoop system must again devise custom
				InputFormat and RecordReader classes in order to facilitate
				distributed image extraction.  The Hadoop Image Processing Framework
				provides this functionality for the extraction task as well, providing
				much greater ease of use for the development of image processing
				applications.
				
				Organizing and specifying the final location of extracted images in a
				large distributed task can be confusing and difficult.  Our framework
				provides this functionality, and allows the user simply to specify
				whether images should be extracted to a local file system or reside on
				the Hadoop DFS.
				
				The process of the Extractor module is explained below:
				
				\textbf{Step 1: Input the image bundle to be extracted.} The image
				bundle specified for extraction is passed as a parameter to the
				Extractor module.  In addition, the user can include an optional
				parameter specifying the images' final location (defaults to local
				file system).  The image bundle is then distributed across the nodes
				as individual map tasks. Each map task will extract the requisite
				images onto the specified filesystem.
				
				\begin{figure}[h]
					\centering
					\includegraphics[width=0.70\textwidth]{ext-map}
					\caption{Individual map task of Extractor Module}
					\label{fig:ext-map}
				\end{figure}
				
				\textbf{Step 2: Split Image bundle across nodes.} The input image
				bundle is split across available nodes using the framework's custom
				input format and record reader classes for maximum throughput. Once a
				map task starts, HImage objects are retrieved.
				
				\textbf{Step 3: Extract individual image.} The image bytes obtained
				from each HImage object are stored onto the filesystem in the
				appropriate file format based on its image type.  The Extractor module
				lacks a reduce phase.
				
				
				
				\chapter{IMAGE PROCESSING ALGORITHMS}
				\label{algorithms}
				To explore the Hadoop Image Processing Framework's capabilities and
				performance, we implemented multiple variations of existing
				widely-used image processing algorithms:
				\begin{itemize}
					\item A stand-alone implementation running on a single node with no
					distributed processing capability
					\item A generic Hadoop implementation which requires a high level of
					software engineering expertise within the Hadoop framework
					\item An implementation that employs the Hadoop Image Processing
					Framework
					\end{itemize}
					
					We chose the following algorithms: Laplacian filtering, Canny edge
					detection and $k$-means image segmentation.  These are widely-used,
					computation-intensive and data-intensive algorithms of varying complexity which
					require large distributed systems for timely operation on hundreds of
					thousands of images.
					
					%\subsection{Flat Field Correction}
					%Flat Field Correction(FFC) is described mathematically below:
					%\begin{equation}
					%	I^{FFC}(x,y) = \frac{I^{RAW}(x,y)-DI(x,y)}{WI(x,y)-DI(x,y)}
					%\end{equation}
					
					%where \begin{math}I^{FFC}(x,y))\end{math} is the flat-field corrected image intensity, \begin{math}DI(x,y)\end{math} is the dark image acquired by closing the camera shutter, \begin{math}I^{RAW}(x,y)\end{math} is the raw uncorrected image density and WI(x,y) is the flat field intensity acquired without any object to correct primarily for spatial shading. This is the simpleset computations that consists of two subtractions and one division for pixel. It needs the DI and WI images co-located from the distributed execution perspective.
					
					\section{Laplacian Filter}
					The Laplacian is a 2-D isotropic measure of the 2nd spatial derivative
					of an image. The Laplacian of an image highlights regions of rapid
					intensity change and is therefore often used for edge detection. The
					Laplacian is often applied to an image that has first been smoothed
					with an approximation of a Gaussian filter in order to reduce
					sensitivity to noise. The operator normally takes a single gray level
					image as input and produces another gray level image as output.
					
					The Laplacian \begin{math} L(x,y)\end{math}of an image with pixel
					intensity values \begin{math}I(x,y)\end{math} is given by
					\begin{equation}
					L(x,y) = \frac{\partial^2 I}{\partial x^2} + \frac{\partial^2 I}{\partial y^2}
					\end{equation} 
					
					This is calculated using a convolution filter.
					
					Since the input image is represented as a set of discrete pixels, we
					have to find a discrete convolution kernel that can approximate the
					second derivatives in the definition of the Laplacian. This is a very
					simple computation that consists of a few additions and
					multiplications.
					
					\section{Canny Edge Detection}
					Canny edge detection\cite{Canny86} is an multi-stage algorithm to
					detect a wide range of edges in images. Edge detection, especially
					step edge detection, is an important technique to extract useful
					structural information from visual information, dramatically reducing
					the amount of data to be processed. Among existing edge detection
					methods, the Canny edge detection algorithm is considered to provide
					reliable edge detection without relying on context-specific
					heuristics.
					
					The process of the Canny edge detection algorithm can defined in five
					stages:
					
					\begin{itemize}
						\item Apply a Gaussian filter to smooth the image and remove noise.
						\item Find the intensity gradients of the image.
						\item Apply non-maximum suppression to get rid of spurious
						responses to edge detection
						\item Apply a double threshold to determine potential edges.
						\item Track edges by hysteresis. Finalize the detection of
						edges by suppressing all edges that are ``weak'' and not
						connected to strong edges.
						\end{itemize}
						
						\section{Image segmentation using $k$-means clustering}
						The $k$-means algorithm is an unsupervised clustering algorithm that
						classifies input data points into multiple classes based on their
						Euclidean distance from each other. The algorithm assumes that the
						data features form a vector space and tries to find natural
						clusterings within it. The points are clustered around
						centroids \begin{math}\mu_{i} \forall i = 1...k\end{math} which are
						obtained by minimizing the objective
						\begin{equation}
						V = \sum\limits_{i=1}^n \sum\limits_{x_{j}\in S_{i}} (x_{j} - \mu_{i})^2
						\end{equation}
						
						where there are \begin{math}k\end{math} clusters \begin{math}
							S_{i},i=1,2,...,k \end{math} and \begin{math} \mu_{i} \end{math}
							is the centroid or mean point of all the points \begin{math}x_{j}
								\in S_{i}\end{math}.
								
								For the purposes of our experiment, we implemented iterative versions
								of the algorithm. The algorithm takes a 2-dimensional image as
								input. The steps of the algorithm are as follows:
								
								\begin{itemize}
									\item Compute the intensity distribution (also called the
									histogram) of the image.
									\item Initialize the centroids with $k$ random intensities.
									\item Repeat the following steps until the cluster labels of
									the image stop changing from one iteration to the next:
									\begin{itemize}
										\item Cluster the points based on the distance of their intensities
										from the centroid's intensity.
										\begin{equation}
										c^{i} := \mbox{argmin}_{j}||x^{i} - \mu_{j}||^2
										\end{equation}
										\item Compute the new centroid for each of the clusters.
										\begin{equation}
										\mu_{i} = \frac{\sum\limits_{i=1}^{m} 1\{c_{i} = j\}x^i}{\sum\limits_{i=1}^{m} 1\{c_{i} = j\}}
										\end{equation}
										
										where \begin{math}k\end{math} is a parameter of the algorithm
										(the number of clusters to be
										found), \begin{math}i\end{math} iterates over all the
										intensities, \begin{math}j\end{math} iterates over all the
										centroids and \begin{math}\mu_{i}\end{math} are the
										centroid intensities.
										\end{itemize}
										\end{itemize}
				
				
				\chapter{EXPERIMENTAL RESULTS}
				We present execution time and programming complexity results for the
				three image processing algorithms described above, demonstrating the
				effectiveness of the Hadoop Image Processing Framework.  These
				algorithms are not particularly challenging to implement on a single
				node, but producing a distributed Hadoop implementation is quite
				challenging and requires a great deal of familiarity with Hadoop
				concepts and distributed software engineering expertise.  On the other
				hand, our framework provides all of the performance and scalability
				benefits of the Hadoop system while retaining the simplicity of the
				single-node implementation.  In this section we describe the dataset, 
				hardware and software specifications used in our experiments, and
				discuss the results.
				
				%\begin{figure}[h]
				%	\centering
				%	\includegraphics[width=0.45\textwidth]{input-canny}
				%	\caption{A Sample Image from downloaded image bundle}
				%	\label{fig:input-canny}
				%\end{figure}
				
				%\begin{figure}[h]
				%	\centering
				%	\includegraphics[width=0.45\textwidth]{output-canny}
				%	\caption{Processed image using canny edge detection}
				%	\label{fig:output-canny}
				%\end{figure}
				
				\section{Characteristics of Image Dataset, Hardware and Software}
				We performed image processing computations on a dataset of about 1 TB
				in size. The dataset was obtained by performing a Flickr search based
				on the keyword ``flight''.  The dataset was extracted using the
				FlickrRipper, an instance of the Ripper interface provided by our
				Hadoop Image Processing Framework.  The framework currently provides
				rippers for Flickr and Google image searches, and more will be
				provided in the future.  A Ripper interface extracts image URLs from
				search results, and can be implemented for any set of results in any
				format.  The experimental dataset is composed of 220,000 images at 4.76
				MB/image, or about 1 TB.
				
				\section{Computer hardware and software characteristics}
				We ran the image processing computations on a 6-node distributed
				cluster and on a single-node desktop computer. The table below
				summarizes the cluster and desktop hardware and software
				specifications. The cluster nodes differ in terms of CPU speed and
				RAM. We installed Hadoop, Cloudera CDH5 and Java 1.7 on the cluster to
				support Java code execution. The desktop computer had a similar
				software configuration to the cluster.
				
				\begin{table}[h]
					\centering
					\caption {Computer hardware and software characteristics.}
					\label{table:table1}
					\begin{tabular}{|l|l|l|l|}
						\hline
						& \multicolumn{1}{c|}{\textbf{Specs}}                            & \multicolumn{1}{c|}{\textbf{Cluster}}                                                                                                           & \multicolumn{1}{c|}{\textbf{Desktop}}                                                                               \\ \hline
						\multirow{2}{*}{Hardware} & Cluster Nodes                                                  & \begin{tabular}[c]{@{}l@{}}6 data nodes with \\ from 2 to 4 virtual \\ processors and 4 GB\\ RAM and name node \\ with 8 GB RAM\end{tabular} & \begin{tabular}[c]{@{}l@{}}Intel Xeon @ 2Ghz\\ 8 cores, 16GB of RAM \\ and hyperthreading\\ activated\end{tabular} \\ \cline{2-4} 
						& Networking                                                     & 1 Gbit/second                                                                                                                                    &                                                                                                                     \\ \hline
						\multirow{3}{*}{Software} & \begin{tabular}[c]{@{}l@{}}Java Virtual\\ Machine\end{tabular} & \begin{tabular}[c]{@{}l@{}}Java version \\ "1.7.0\_45"\\ Java SE Runtime\\ Environment\end{tabular}                                             & \begin{tabular}[c]{@{}l@{}}Java version \\ "1.7.0\_45"\\ Java SE Runtime\\ Environment\end{tabular}                 \\ \cline{2-4} 
						& Hadoop                                                         & \begin{tabular}[c]{@{}l@{}}Hadoop 2.5\\ Cloudera CDH5\end{tabular}                                                                              &                                                                                                                     \\ \cline{2-4} 
						& \begin{tabular}[c]{@{}l@{}}Operating\\ System\end{tabular}     & CentOS 6.5                                                                                                                                      & CentOS 6.5                                                                                                          \\ \hline
					\end{tabular}
					
					
				\end{table}
				
				
				The desktop implementations of the algorithms are Java programs for
				processing a small image dataset. As soon as the data set is processed
				the program halts. The desktop implementation starts as a single
				thread, and images are processed one at a time in succession.  When
				the dataset size is low, each program runs smoothly and with excellent
				performance.  However, as the size of the data to be processed reaches
				the terabyte scale, such a single-process approach fails utterly.
				Data on this scale must be processed in distributed fashion using
				Hadoop.
				
				Each algorithm is also implemented on a Hadoop cluster without using
				the Hadoop Image Processing Framework.  These implementations overcome
				the problem of handling large datasets but they suffer from extremely
				complex code.  The user needs to write a great deal of code and
				understand the technical details of Hadoop. Writing such code,
				including custom InputFormats and RecordReaders, is cumbersome and
				error-prone. Figure \ref{fig:module-chart} compares the actual lines of
				code a user needs to write in order to run our experimental algorithms
				(or any similar image processing application) on a Hadoop cluster.
				
				Finally, we compare the same algorithms implemented on a Hadoop
				cluster using the framework.  These implementations overcome the
				problems of handling large datasets and managing interoperability
				between image datatypes while holding code complexity down to the same
				level as the single-process, local implementation.  Users enjoy
				automated creation of custom InputFormats and RecordReaders, and can
				tune settings for the desired output.  The framework is so transparent
				that the user needs only a few lines of code to process a large image
				data set. It hides the technical details of MapReduce and runs the
				code in highly parallelized fashion.  The code required to implement
				each image processing algorithm is no more complex than the same
				algorithm implemented on a local, single-threaded system, but the
				performance improvement for large datasets is immense.
				
				\begin{table}[h]
					\centering
					\caption{Characteristics of computational elasticity, coding complexity and data locality for the three experimental platforms.}
					\label{table:table2}
					\begin{tabular}{|l|l|l|l|}
						\hline
						\multicolumn{1}{|c|}{\textbf{\begin{tabular}[c]{@{}c@{}}Computa-\\ tional\\ Platform\end{tabular}}} & \multicolumn{1}{c|}{\textbf{\begin{tabular}[c]{@{}c@{}}Computational \\ Elasticity\end{tabular}}}      & \multicolumn{1}{c|}{\textbf{\begin{tabular}[c]{@{}c@{}}Code \\ Complexity\end{tabular}}}                                                                                                                                                  & \multicolumn{1}{c|}{\textbf{\begin{tabular}[c]{@{}c@{}}Data \\ Locality\end{tabular}}}                                                                           \\ \hline
						Desktop                                                                                             & \begin{tabular}[c]{@{}l@{}}Low : limited by \\ RAM and CPU of \\ the executing\\ computer\end{tabular} & \begin{tabular}[c]{@{}l@{}}Normal : Complexity \\ lies in writing the \\ algorithm\end{tabular}                                                                                                                                           & \begin{tabular}[c]{@{}l@{}}All data are \\ on local disk\end{tabular}                                                                                            \\ \hline
						\begin{tabular}[c]{@{}l@{}}Hadoop:\\ without \\ using \\ framework\end{tabular}                     & \begin{tabular}[c]{@{}l@{}}High: Nodes can\\  be requested \\ as needed\end{tabular}                   & \begin{tabular}[c]{@{}l@{}}Heavy : Complexity \\ lies in writing every \\ module in framework \\ including custom \\ InputFormats and \\ RecordReaders.  User \\ requires high \\ technical expertise \\ with Hadoop.\end{tabular} & \begin{tabular}[c]{@{}l@{}}All data resides\\ in HDFS. Custom\\ InputFormats \\ need to be \\ created to launch\\ computations \\ in appropriate \\ locations\end{tabular} \\ \hline
						\begin{tabular}[c]{@{}l@{}}Hadoop:\\ using \\ framework\end{tabular}                                & \begin{tabular}[c]{@{}l@{}}High: Nodes can \\ be requested \\ as needed\end{tabular}                   & \begin{tabular}[c]{@{}l@{}}Normal : Complexity \\ lies in writing \\ algorithm and is \\ equivalent to writing \\ on desktop platform\end{tabular}                                                                                        & \begin{tabular}[c]{@{}l@{}}All data resides \\ in HDFS. Highly \\ parallelized.\end{tabular}                                                                     \\ \hline
					\end{tabular}
				\end{table}
				
				
				
				
				\iffalse
				\begin{table}[h]
					\centering
					\begin{tabular}{|c|c|c|}
						\hline
						\textbf{Module} & \textbf{\begin{tabular}[c]{@{}c@{}}SLOC \\ using framework\end{tabular}}                                                          & \textbf{\begin{tabular}[c]{@{}c@{}}SLOC \\ without framework\end{tabular}}                                                                                    \\ \hline
						Downloader      & 3 - 5 lines                                                                                                                       & \begin{tabular}[c]{@{}c@{}}300 - 1000 lines.\\ Custom Input Formats\\ and RecordReaders are\\ to be implemented\end{tabular}                                  \\ \hline
						Processor       & 3 - 5 lines                                                                                                                       & \begin{tabular}[c]{@{}c@{}}400 - 1000 lines\\ Many overheads. \\ Converting to image \\ data types, Mapper \\ locality should be taken\\ care of\end{tabular} \\ \hline
						Extractor       & 2 - 8 lines                                                                                                                       & \begin{tabular}[c]{@{}c@{}}200 - 1000 lines\\ Depends on the extraction\\ process.\end{tabular}                                                               \\ \hline
						Algorithm       & \begin{tabular}[c]{@{}c@{}}200 - 300 lines\\ adds a few lines overhead\\ to automate the process\\ for MapReduce job\end{tabular} & 200 - 250 lines                                                                                                                                               \\ \hline
					\end{tabular}
				\end{table}
				\fi
				
				\section{Observations}   
				
				We have explored the space of image processing algorithms, applying
				different hardware and software configurations to the various data
				sets.  Specifically, we compared:
				
				\begin{itemize}
					\item Three image processing algorithms described in Section
					\ref{algorithms}
					\item Datasets of varying image counts
					\item Configurations of hardware cluster used for computations 
					\item Coding complexity - measured in Lines of Code (LOC)
				\end{itemize}
				
				The numerical results are shown in Figures \ref{fig:perf-chart}
				through \ref{fig:files-chart}.
				
				Figure \ref{fig:perf-chart} illustrates the performance comparison of
				Canny edge detection on a single node, on Hadoop using our framework
				and Hadoop without using the framework.
				
				\begin{figure}[h]
					\centering
					\includegraphics[width=0.80\textwidth]{perf-graph}
					\caption{Canny edge detection computation executed on a single node
						and on a Hadoop cluster with and without using framework}
					\label{fig:perf-chart}
				\end{figure}
				
				It can be observed that the Hadoop cluster performance is nearly
				identical whether the Hadoop Image Processing Framework is used or
				not.  The single node performance is satisfactory on small input
				sizes; in fact it outperforms Hadoop when the amount of data is very
				small owing to the lack of MapReduce overhead.  However, as the amount
				of data grows the single node performance becomes increasingly poor.
				The same pattern is observed with any image processing algorithm.
				
				Figure \ref{fig:module-chart} compares the coding complexity of the
				different image handling and processing tasks described in Section
				\ref{methodology}, for the Canny edge detection task. The Downloader
				module and Extractor module are simple to use and require few lines of
				code to configure. The processor module includes the specific
				algorithm devised by the user, which can be significant amount of
				code, but using our framework to configure the algorithm for MapReduce
				requires almost no additional coding.
				
				\begin{figure}[h]
					\centering
					\includegraphics[width=0.80\textwidth]{module-chart2}
					\caption{Comparing coding complexities of modules in different
						environments}
					\label{fig:module-chart}
				\end{figure}
				
				Figure \ref{fig:comp-chart} shows the coding complexity of the
				processor portion of our three experimental image processing
				algorithms.  The naive single-process, single-node algorithm requires
				a similar amount of code to using our highly parallel, distributed
				framework, while direct implementation of the algorithm using Hadoop
				and MapReduce requires a great deal more effort.  The Hadoop Image
				Processing Framework hides all of this complexity from the user.
				
				\begin{figure}[h]
					\centering
					\includegraphics[width=0.90\textwidth]{comp-chart}
					\caption{Comparing coding complexity of Laplacian filter,
						Canny edge detection and $k$-means clustering in different
						environments}
					\label{fig:comp-chart}
				\end{figure}
				
				The Hadoop Image Processing Framework provides transparency on
				multiple levels.  Users can use the predefined MapReduce modules
				(Downloader, Processor and Extractor) for processing the image
				modules, and most image processing tasks require no more than this.
				Users may also write custom MapReduce tasks within the framework,
				taking advantage of its hierarchical construction.  This still
				protects the user from the full complexity and error-prone nature of
				the full Hadoop framework, but requires additional knowledge and
				expertise on the user's part.
				
				The Hadoop Image Processing Framework is intended to be extremely
				simple to use. The framework strictly adheres to Java file writing and
				reading techniques, extending those conventions to the Hadoop
				framework's notion of reading and writing bundle files.  In this way,
				working with Hadoop image processing is exactly like working on a
				single system.  The main aim of the framework is to provide useful
				software abstractions such that programming on a Hadoop cluster is
				equivalent to programming on a single computer.  Listing 1
				demonstrates the similarity between Java code written for a single
				machine and the same code using the image processing framework.  The
				framework's use of ordinary Java conventions helps in understanding
				the software engineering process and reduces complexity.  Listing 2
				demonstrates the mechanism for setting the image headers of processed
				images.
				
				\begin{figure}[h]
					\centering
					\includegraphics[width=0.90\textwidth]{files-chart2}
					\caption{Comparing file writers and readers with and without using
						framework}
					\label{fig:files-chart}
				\end{figure}
				\newpage
				\begin{lstlisting}[caption = Comparing FileWriter instance in java and SequenceBundleWriter instance in Hadoop Image Processing framework ]
				
				//Sample File Writing in Java
				File file = new File("test");
				FileWriter fw = new FileWriter(file);
				fw.append(val)
				fw.close();
				
				//Sample BundleFile Writing in Hadoop using Framework
				BundleFile file = new BundleFile("test_bundle.seq");
				SequenceBundleWriter sbw = new SequenceBundleWriter(file);
				sbw.append(himage);
				sbw.close();
				\end{lstlisting}
				
				\begin{lstlisting}[caption = Setting image headers for processed images using Hadoop Image Processing Framework]
				
				//input - HImage is sent as input 
				CannyEdge canny = new CannyEdge(input);
				canny.process();
				HImage himage = canny.getProcessedImage();
				himage.setImageHeader(input.getImageHeader());
				
				\end{lstlisting}
				
				\chapter{FUTURE WORK}
				This paper has described our Hadoop Image Processing Framework for
				implementing large scale image processing applications.  The framework
				is designed to abstract the technical details of Hadoop's powerful
				MapReduce system and provide an easy mechanism for users to process
				large image datasets. We provide software machinery for storing images
				in the various Hadoop file formats and efficiently accessing the Map
				Reduce pipeline.  By providing interoperability between different
				image data types we allow the user to leverage many different
				open-source image processing libraries. Finally, we provide the means
				to preserve image headers throughout image manipulation process,
				retaining useful and valuable information for image processing and
				vision applications.
				
				With these features, the framework provides a new level of
				transparency and simplicity for creating large-scale image processing
				applications on top of Hadoop's MapReduce framework.  We demonstrate
				the power and effectiveness of the framework in terms of performance
				enhancement and complexity reduction.  The Hadoop Image Processing
				Framework should greatly expand the population of software developers
				and researchers easily able to create large-scale image processing
				applications.
				
				\chapter{CONCLUSION}
				In the near future we hope to extend the framework into a full-fledged
				multimedia processing framework. We would like to improve the
				framework to handle audio and video processing over Hadoop with
				similar ease. We also intend to add a CUDA module to allow processing
				tasks to make use of machines' graphics cards. Finally, we intend to
				develop our system into a highly parallelized open-source Hadoop
				multimedia processing framework, providing web-based graphical user
				interfaces for image processing applications.
				
					\bibliography{references}
					\bibliographystyle{abbrv}
				
				
				
				
				
				
				
				
				
				
				
				
				%HELLO EORLD
              % BIBLIOGRAPHY
 %             \bibliography{references.bib}
				
              % APPENDICES
              %\appendix
               % (Recommend storing text in separate *.tex file, then
                      % including with \include{<filename>}.)

              % MISCELLANEOUS
              \begin{vita}{Sridhar Vemula}{Master of Science}{Computer Science} % Creates vita
                %\vitaitem{Personal Data:} <Text> % OPTIONAL
                
                \vitaitem{\renewcommand\labelitemi{}\begin{itemize}
                		\item {Education:}
                		\item{Completed the requirements for the Master of Science in Computer Science at
                			Oklahoma State University, Stillwater, Oklahoma in May, 2015.}
                	\end{itemize}}
               % \vitaitem{Experience:} 
               % \vitaitem{Professional Memberships:} <Text>
              \end{vita}

             
				
              \end{document}