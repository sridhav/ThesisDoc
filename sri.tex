\documentclass{osuthesis}
\usepackage{graphicx}
\usepackage{multirow}

\usepackage{listings}
\usepackage{color}

\definecolor{dkgreen}{rgb}{0,0.6,0}
\definecolor{gray}{rgb}{0.5,0.5,0.5}
\definecolor{mauve}{rgb}{0.58,0,0.82}

\lstset{
	language=Java,
	aboveskip=3mm,
	belowskip=3mm,
	showstringspaces=false,
	columns=flexible,
	basicstyle={\small\ttfamily},
	numbers=none,
	numberstyle=\tiny\color{gray},
	keywordstyle=\color{blue},
	commentstyle=\color{dkgreen},
	stringstyle=\color{mauve},
	breaklines=true,
	breakatwhitespace=true,
	tabsize=3
	}


\graphicspath{ {images/} }
              \title{HADOOP IMAGE PROCESSING FRAMEWORK} % Title in ALL CAPS
             \formattedtitle{HADOOP IMAGE PROCESSING FRAMEWORK} % Same as \title, but with line
                                       % breaks to obtain the inverted
                                      % pyramid shape required for
                                      % title and approval pages
              \author{SRIDHAR VEMULA} % Author in ALL CAPS

             % Earned degree(s)
              \degreeone{\\%
               Master of Science in Computer Science\\%
               Oklahoma State University\\%
               Stillwater, OK\\%
               2015}
% %             \degreeone{%
%               <Degree>\\%
 %               <Institution>\\%
  %              <City>, <State/Country>\\%
   %             <Year>} % (If applicable)

              % Current degree information
              \degreesought{Master of Science} % Degree in ALL CAPS
              \degreedate{May, 2015}
              \majorfield{Computer Science}

              \begin{document}
             \bibliographystyle{plain}  % Several styles available,
                                         % check with your department

             % FOREMATTER
              \maketitle        % Creates title page.
              \makeapproval{3}  % Creates spproval page.  The argument
                                % (the numeral 4,5, or 6) is the number of
                               % signatures required (Remember to add one
                                % for the Dean of the Graduate College.)
%             \begin{preface}   % Creates Preface page (OPTIONAL)
             %   <Preface text here.>
 %             \end{preface}

  %            \begin{acknowledge}  % Creates Acknowledgments
                                   % page (OPTIONAL)
               % <Acknowledgment text here.>
   %          \end{acknowledge}
			 \begin{abstract}{SRIDHAR VEMULA}{MASTER IN SCIENCE}{COMPUTER SCIENCE} % Creates abstract
			              \par With the rapid growth of social media, the number of images being
			              uploaded to the internet is exploding. Massive quantities of images
			              are shared through multi-platform services such as Snapchat,
			              Instagram, Facebook and WhatsApp; recent studies estimate that over
			              1.8 billion photos are uploaded every day.  However, for the most
			              part, applications that make use of this vast data have yet to
			              emerge. Most current image processing applications, designed for
			              small-scale, local computation, do not scale well to web-sized
			              problems with their large requirements for computational resources and
			              storage.  The emergence of processing frameworks such as the Hadoop
			              MapReduce\cite{dean2008} platform addresses the problem of providing
			              a system for computationally intensive data processing and distributed
			              storage. However, to learn the technical complexities of developing
			              useful applications using Hadoop requires a large investment of time
			              and experience on the part of the developer.  As such, the pool of
			              researchers and programmers with the varied skills to develop
			              applications that can use large sets of images has been limited. To
			              address this we have developed the Hadoop Image Processing Framework,
			              which provides a Hadoop-based library to support large-scale image
			              processing. The main aim of the framework is to allow developers of
			              image processing applications to leverage the Hadoop MapReduce
			              framework without having to master its technical details and introduce
			              an additional source of complexity and error into their programs.

			              \end{abstract}
              \tableofcontents  % Creates table of contents
              \listoftables
             \listoffigures    % Creates list of figures (IF APPLICABLE)
	       
%              \listoftables     % Creates list of tables (IF APPLICABLE)
			
              % BODY OF THESIS
              \include{body2}  % (Recommend storing text in separate *.tex file, then
                      % including with \include{<filename>}.)
				
              % BIBLIOGRAPHY
 %             \bibliography{references.bib}
				
              % APPENDICES
              \appendix
               % (Recommend storing text in separate *.tex file, then
                      % including with \include{<filename>}.)

              % MISCELLANEOUS
              \begin{vita}{Sridhar Vemula}{Master of Science}{Computer Science} % Creates vita
                %\vitaitem{Personal Data:} <Text> % OPTIONAL
                
                \vitaitem{\renewcommand\labelitemi{}\begin{itemize}
                		\item {Education:}
                		\item{Completed the requirements for the Masters degree with a major in Computer Science at
                			Oklahoma State University in May, 2015.}
                	\end{itemize}}
               % \vitaitem{Experience:} 
               % \vitaitem{Professional Memberships:} <Text>
              \end{vita}

             
				
              \end{document}